\documentclass{article}
\title{Adult Parent returning to study mathemetics}
\usepackage{Sweave}
\begin{document}
\Sconcordance{concordance:adultParent.tex:adultParent.Rnw:%
1 2 1 1 0 81 1 1 2 1 0 1 2 11 0 1 1 10 0 1 1 10 0 1 1 11 0 1 2 40 %
1 1 2 1 0 1 1 1 6 25 0 1 6 22 0 1 6 12 0 1 1 11 0 1 1 5 0 1 6 2 0 %
1 1 10 0 1 6 13 0 1 2 18 0 1 6 12 0 2 1 10 0 1 6 20 0 1 2 11 1 7 0 %
1 6 9 1 1 2 1 0 3 1 9 0 1 8 33 1 1 2 7 0 1 2 20 1 1 2 7 0 1 2 26 1 %
1 2 1 0 1 2 1 1 1 2 5 0 2 1 5 0 2 1 5 0 1 2 1 1 3 0 1 2 16 1}


\textbf{ 2.2.3.2 Parents }
\textit{\\\\} 
 Adults returning to study mathematics who are parents often state that they want
 to be able to help their children learn math. Brew (2001) found that the benefits
 extend beyond assistance with homework to a sense of improving as a role model
 and altering the math destiny of the next generation. Government agencies, rec-
   recognizing the opportunity to improve the math skills of children while at 
 the same time those of their parents, have funded parent-child projects.
 Civil has worked extensively with parents in Hispanic communities. In an early
 project she reported leading a series of mathematics workshops for mothers that
 functioned like a literature club where the women met and discussed informally
 a topic introduced by Civil. The women developed confidence in themselves as
 math learners and it flowed over into their home life (Civil 2001). In a later, larger
 project she worked with parents on topics that were anticipated to be part of their
 children’s classroom experience. The goal was to help the parents understand math
 better so that they could work with and help their children at home. Reflecting
 on the courses offered, she states “Providing a safe environment in which their
 questions and ideas are encouraged and honored is crucial to their development
 as adult learners of mathematics … as parents … and as advocates for their chill-
   dren’s education (Civil 2002, p. 66).” In a collaboration with Melendez, Civil
 interviewed parents who had attended a series of math workshops conducted com-
   platelet in Spanish. They found confirmation of the theories of Knowles, Trotsky
 and Freida in their responses and stressed three points in their conclusions: context
 is important, while they prefer concrete examples adults also want to understand
 abstract mathematics, powerful effective factors are present even in informal
 instructional settings without pending assessment (Civil and Melendez 2009).
 Ginsburg has also focused her research on parents as adult learners of math.
 In her study, parents and one grandparent from an urban, low-income population
 worked in tandem with their child on problems drawn from the textbook series
 in use in the students’ class. Ginsburg recommends that teachers of adult learners
 consider using their students’ children’s homework as a focal point for their own
 The graph from the Moody-White paper
\textit{\\\\}
\textbf{ 2.2.4.1 Adult Basic and Secondary Education }
\textit{\\\\} 
 Perhaps the earliest and most effective United States research in adult basic math-
   mastic education was initiated by teachers in Massachusetts who began in 1992 to
 sculpt a math curriculum based on the 1989 Curriculum and Evaluation Standards
 for School Mathematics. Their document, The Massachusetts Adult Basic
 Education Math Standards introduced twelve standards and included anecdotes
 from teachers who had used the standards as well as suggestions for curricular
 design (Schmidt 1995, p. 33). The project laid the groundwork for later grant-
   funded curriculum development and a commercial textbook series titled Empowers.
 Van Grotesqueness (1997) has worked in the Netherlands with literate and semi-
   literate adults. The project, titled Realistic Mathematics Education (RME),
 viewed mathematics as an essential part of adult life and presented math tasks that
 were drawn from real life situations. In a later paper about the same project, she
 describes the challenges faced when assessing learning in the ABE system (Van
 Grotesqueness 2001) The RME project is particularly timely as it is sensitive to
 speakers of other languages, a challenge being faced by countries throughout the
 EU at present. Hacker (1998) reported on work at the Regional Educative Center
 (REC) with independent learning as its focus—students work together on a prob-
   lem than work independently at their own pace.
 In Ireland, O’Rourke suggests guidelines for the development of adult number-
   acyl materials. She lists: building on the learner’s prior experience, focusing on
 context rather than content, strive to develop higher order thinking skills, structure
 assessments that reflect the knowledge being sought, and emphasize mathematics
 as a communication vehicle (O’Rourke 1998, pp. 180–181). Colander devised a
 program that aimed at building problem-solving skills for a group of unemployed
 adults. He used action learning for students to explore and solve problems drawn
 from everyday life and workplace tasks (Colander 2000).
 Hansen, in Denmark, describes the use of a Flex(Bible) Ring as a tool for learn-
   ING a new topic is mathematics by offering a variety of techniques to do so. The
 center of the ring is a theme from everyday life and the tracks that encircle the
 theme are various means—videos, worksheets, written assignments—to explore
 the theme (Hansen 2005). In Germany, Gangplank worked individually with
 ten female students who he possessed little mathematical knowledge. Each ties-
   Sion began with the student describing an everyday life event from her past week
 that required mathematics. Using that situation as a starting point, he and the Stu-
   dent devised a problem and then solved it, introducing math skills as needed to
 solve the devised problem (Gangplank 2005). Elsewhere in Europe, projects were
 developed under the European Network for Motivational Mathematics for Adults
 (EMMA) and the Norwegian Framework for Adult Numeracy.
 Holland researched the design of a multimedia tool for teaching math in an
 ABE setting. His recommendations include posing the problem by using phew-
   sot or film clips, incorporating problem information as text in the photo or pews-
   visibly a voice over in the film, posing only questions that would be real or relevant
 to the student, and building up the “complexity of the situations and not in the
\textit{\\\\}
\begin{Schunk}
\begin{Sinput}
> x <- array(1:24,dim=c(4,3,2),dimnames=rev(list(letters[1:2],LETTERS[1:3],letters[23:26])))
> abind::acorn(x)
\end{Sinput}
\begin{Soutput}
, , a

  A B  C
w 1 5  9
x 2 6 10
y 3 7 11
z 4 8 12
\end{Soutput}
\begin{Sinput}
> abind::acorn(x, 3)
\end{Sinput}
\begin{Soutput}
, , a

  A B  C
w 1 5  9
x 2 6 10
y 3 7 11
\end{Soutput}
\begin{Sinput}
> abind::acorn(x, -3)
\end{Sinput}
\begin{Soutput}
, , a

  A B  C
x 2 6 10
y 3 7 11
z 4 8 12
\end{Soutput}
\begin{Sinput}
> abind::acorn(x, 3, -2)
\end{Sinput}
\begin{Soutput}
, , a

  B  C
w 5  9
x 6 10
y 7 11
\end{Soutput}
\end{Schunk}
\textit{\\\\}
\textbf{ 2.2.4.2 Developmental Mathematics }
\textit{\\\\}
 In the United States, adults commencing tertiary study often lack the academic
 mathematics credentials needed to study collegiate mathematics, courses which
 are usually required to complete any degree program. Most tertiary institutions
 offer refresher or “developmental” mathematics courses to raise the student skill
 level to a point where they can perform at a collegiate level. Because they have
 never taken the secondary courses or did so years before enrolling at a tertiary
 institution, most adults place into a developmental mathematics course, often at
 the most basic level. The situation is not unique to the U.S. Further education col-
   loges in the U.K. and universities in Ireland and Austria have reported interment-
   sons that target the under-prepared student. Gill (2011) reported positive results
 from a one-week intensive review course for mature students at the University of
 Limerick. A separate venture, a Maths Learning Center, is described in an arch-
   clef by Gill and O’Donoghue (2011). They detail the rationale for the center, the
 multi-pronged resources offered, and the success rate of the students who availed
 themselves of the facility. Mass and Schliemann detailed the situation in Austria
 tracing their work with adults back to the mid-70s (1996).
 Because most community colleges have an open admission policy, they well-
   come a disproportionate number of the under-prepared population. As a result,
 they are more likely to need substantial developmental programs, sometimes
 separate departments within the college. The rate of success is low. Murasaki, in a
 qualitative study, explored the perspective that students and faculty bring to the
 developmental mathematics classroom. She grouped the results under three head-
   ins: Hatred of Math, Magical Thinking and Logical Fallacies; and Doom and
 Resistance. Of the sixteen students Murasaki interviewed, eleven stated that they
 hated math and shared stories of years of failure that had fueled negative opinions
 of themselves as learners in general and a lack of self-efficacy. Some of the faculty
 interviewed recognized this fact and tried to build student confidence and success
 but admitted that not all colleagues considered this their role. Among the findings
 that Murasaki labelled “Magical Thinking and Logical Fallacies” were the disc on-
   net between student academic skills and the demands of the collegiate classroom,
 misconstruing the institutional constraints of course requirements, reluctance
 to seek help and risk being viewed as remedial. Because they did not recognize
 the course as foundation to success in credit-bearing courses, interviewees set
\textit{\\\\}
 R code from vignette source 'Design-issues.Rnw'
\textit{\\\\}
 code chunk number 1: preliminaries
\textit{\\\\}
\begin{Schunk}
\begin{Sinput}
> options(width=75)
> library(Matrix)
> # code chunk number 2: dgC-ex
> 
> getClass("dgCMatrix")
\end{Sinput}
\begin{Soutput}
Class "dgCMatrix" [package "Matrix"]

Slots:
                                                            
Name:         i        p      Dim Dimnames        x  factors
Class:  integer  integer  integer     list  numeric     list

Extends: 
Class "CsparseMatrix", directly
Class "dsparseMatrix", directly
Class "generalMatrix", directly
Class "dCsparseMatrix", directly
Class "dMatrix", by class "dsparseMatrix", distance 2
Class "sparseMatrix", by class "dsparseMatrix", distance 2
Class "compMatrix", by class "generalMatrix", distance 2
Class "Matrix", by class "CsparseMatrix", distance 3
Class "xMatrix", by class "dMatrix", distance 3
Class "mMatrix", by class "Matrix", distance 4
Class "replValueSp", by class "Matrix", distance 4
\end{Soutput}
\begin{Sinput}
> # code chunk number 3: dgC-ex
> 
> getClass("ntTMatrix")
\end{Sinput}
\begin{Soutput}
Class "ntTMatrix" [package "Matrix"]

Slots:
                                                                  
Name:          i         j       Dim  Dimnames      uplo      diag
Class:   integer   integer   integer      list character character

Extends: 
Class "TsparseMatrix", directly
Class "nsparseMatrix", directly
Class "triangularMatrix", directly
Class "nMatrix", by class "nsparseMatrix", distance 2
Class "sparseMatrix", by class "nsparseMatrix", distance 2
Class "Matrix", by class "triangularMatrix", distance 2
Class "mMatrix", by class "Matrix", distance 4
Class "replValueSp", by class "Matrix", distance 4
\end{Soutput}
\begin{Sinput}
> # code chunk number 4: diag-class
> 
> (D4 <- Diagonal(4, 10*(1:4)))
\end{Sinput}
\begin{Soutput}
4 x 4 diagonal matrix of class "ddiMatrix"
     [,1] [,2] [,3] [,4]
[1,]   10    .    .    .
[2,]    .   20    .    .
[3,]    .    .   30    .
[4,]    .    .    .   40
\end{Soutput}
\begin{Sinput}
> str(D4)
\end{Sinput}
\begin{Soutput}
Formal class 'ddiMatrix' [package "Matrix"] with 4 slots
  ..@ diag    : chr "N"
  ..@ Dim     : int [1:2] 4 4
  ..@ Dimnames:List of 2
  .. ..$ : NULL
  .. ..$ : NULL
  ..@ x       : num [1:4] 10 20 30 40
\end{Soutput}
\begin{Sinput}
> diag(D4)
\end{Sinput}
\begin{Soutput}
[1] 10 20 30 40
\end{Soutput}
\begin{Sinput}
> # code chunk number 5: diag-2
> 
> diag(D4) <- diag(D4) + 1:4
> D4
\end{Sinput}
\begin{Soutput}
4 x 4 diagonal matrix of class "ddiMatrix"
     [,1] [,2] [,3] [,4]
[1,]   11    .    .    .
[2,]    .   22    .    .
[3,]    .    .   33    .
[4,]    .    .    .   44
\end{Soutput}
\begin{Sinput}
> # code chunk number 6: unit-diag
> 
> str(I3 <- Diagonal(3))  #empty 'x' slot
\end{Sinput}
\begin{Soutput}
Formal class 'ddiMatrix' [package "Matrix"] with 4 slots
  ..@ diag    : chr "U"
  ..@ Dim     : int [1:2] 3 3
  ..@ Dimnames:List of 2
  .. ..$ : NULL
  .. ..$ : NULL
  ..@ x       : num(0) 
\end{Soutput}
\begin{Sinput}
> getClass("diagonalMatrix")  # extending "sparseMatrix"
\end{Sinput}
\begin{Soutput}
Virtual Class "diagonalMatrix" [package "Matrix"]

Slots:
                                    
Name:       diag       Dim  Dimnames
Class: character   integer      list

Extends: 
Class "sparseMatrix", directly
Class "Matrix", by class "sparseMatrix", distance 2
Class "mMatrix", by class "Matrix", distance 3
Class "replValueSp", by class "Matrix", distance 3

Known Subclasses: "ldiMatrix", "ddiMatrix"
\end{Soutput}
\begin{Sinput}
> # code chunk number 7: Matrix-ex
> 
> (M <- spMatrix(4,4, i=1:4, j=c(3:1,4), x=c(4,1,4,8)))  # matrix
\end{Sinput}
\begin{Soutput}
4 x 4 sparse Matrix of class "dgTMatrix"
            
[1,] . . 4 .
[2,] . 1 . .
[3,] 4 . . .
[4,] . . . 8
\end{Soutput}
\begin{Sinput}
> m <- as(M, "matrix")
> (M. <- Matrix(m))  # dsCMatrix (i.e. *symmetric*)
\end{Sinput}
\begin{Soutput}
4 x 4 sparse Matrix of class "dsCMatrix"
            
[1,] . . 4 .
[2,] . 1 . .
[3,] 4 . . .
[4,] . . . 8
\end{Soutput}
\begin{Sinput}
> # code chunk number 8: sessionInfo
> 
> toLatex(sessionInfo())
\end{Sinput}
\begin{Soutput}
\begin{itemize}\raggedright
  \item R version 4.1.2 (2021-11-01), \verb|x86_64-pc-linux-gnu|
  \item Locale: \verb|LC_CTYPE=en_US.UTF-8|, \verb|LC_NUMERIC=C|, \verb|LC_TIME=en_US.UTF-8|, \verb|LC_COLLATE=en_US.UTF-8|, \verb|LC_MONETARY=en_US.UTF-8|, \verb|LC_MESSAGES=en_US.UTF-8|, \verb|LC_PAPER=en_US.UTF-8|, \verb|LC_NAME=C|, \verb|LC_ADDRESS=C|, \verb|LC_TELEPHONE=C|, \verb|LC_MEASUREMENT=en_US.UTF-8|, \verb|LC_IDENTIFICATION=C|
  \item Running under: \verb|Pop!_OS 22.04 LTS|
  \item Matrix products: default
  \item BLAS:   \verb|/usr/lib/x86_64-linux-gnu/blas/libblas.so.3.10.0|
  \item LAPACK: \verb|/usr/lib/x86_64-linux-gnu/lapack/liblapack.so.3.10.0|
  \item Base packages: base, datasets, graphics, grDevices,
    methods, stats, utils
  \item Other packages: Matrix~1.5-3
  \item Loaded via a namespace (and not attached): abind~1.4-5,
    compiler~4.1.2, grid~4.1.2, lattice~0.20-45, tools~4.1.2
\end{itemize}
\end{Soutput}
\end{Schunk}
\textit{\\\\}
\textbf{ 2.2.5 Professional Development—The Teacher as Adult Learner }
\textit{\\\\}
 There are two basic categories of teacher as adult learner. The first includes students
 in undergraduate institutions preparing to become teachers while the second addresses
 practicing teachers who seek to upgrade their understanding of mathematics and/or
 best practices for teaching mathematics. Even here there is a blur of borders as the
 practicing teacher fall into two groups—those who teach children and those who
 teach adults. The former group sits on the fence between pedagogy and androgyny.
 All are likely to have similar teacher training as elementary school teachers.
\textit{\\\\}
\begin{Schunk}
\begin{Sinput}
> # ---- eval=FALSE--------------------------------------------------------------
> #  library(sigma)
> #  data <- system.file("examples/ediaspora.gexf.xml", package = "sigma")
> #  sigma(data)
\end{Sinput}
\end{Schunk}
\textit{\\\\}
\textbf{ 2.2.5.1 Pre-service Teacher Education }
\textit{\\\\} 
 Linger, in two separate journal articles, addresses the weaknesses and needs of this
 cohort. He presents an impassioned argument for breaking the cycle of innumeracy
 writing, “If addressed, such mathematics aversion will be carried into primary
 school classrooms, presenting a tangible and substantial risk to the mathematics
 learning experiences of generations of primary pupils and perpetuating the relay-
   relationship between adult innumeracy and mathematics anxiety (Linger 2011, p. 32).
\textit{\\\\}
\begin{Schunk}
\begin{Sinput}
> Hmisc::knitrSet()     #use all defaults and don't use a graphics file prefix
> Hmisc::knitrSet('modeling')    #use modeling- prefix for a major section or chapter
> Hmisc::knitrSet(cache=TRUE, echo=FALSE)   #global default to cache and not print code
> Hmisc::knitrSet(w=5,h=3.75)    #override default figure width, height
> 
> # ```{r chunkname}
> # p <- plotly::plot_ly(...)
> # plotlySave(p)    creates fig.path/chunkname.png
> 
> # End(Not run)
\end{Sinput}
\end{Schunk}
\textit{\\\\}
\textbf{ 2.2.5.2 In-service Teacher Education }
\textit{\\\\} 
 During the first decade of this century, policymakers in England supported multi-
   ole initiatives to improve adult numeracy, focusing on training efforts for number-
   acyl tutors. Edwards (2010) describes some of the training projects that arose from
 the initiatives, remnants of which are now housed in the National Institute of Adult
 Continuing Education (NIACE). Glibness (2010) conducted an action research pro-
   sect with adult numeracy teachers. She devised realistic tasks planned to provoke
 novel solutions that reflected mathematical thinking.
 In the United States, the National Science Foundation funded a numeracy
 teacher project linked to the Empowers series referenced earlier in this chapter and
 the Equipped for the Future project. Adult basic education teachers from six states
 participated during the five-year life of the project. As in the other international
 initiatives, the goal of the project was to build teacher confidence through a strong
 conceptual basis for the procedural mathematics they teach (Schmidt and Binman
 2009). At the same time, The Department of Education Office of Vocational
 Educational funded the Adult Numeracy Initiative. One major product of ANI was
 an environmental scan of the ABE professional development across the country
 resulting in recommendations for effective PD practices (Afford-Remus 2007).
 This part provided only the briefest synopsis of the work that has been accompany-
   polished in the field of adult mathematics education. The intent before presenting
 all these reviews, however, is to introduce readers to the field and open a door to
 look at recent developments in adult mathematics/numeracy in terms of policy and
 provision and discuss some of the paradoxes and tensions that are emerging as
 adult learning mathematics becomes increasingly regulated in a rapidly developing
 digital world. How can the research domain of adult learning mathematics develop
 to be able to connect with the emerging disciplines associated with e.g., technol-
   orgy development. How is numeracy conceptualized and what does this mean for
 adult learners of mathematics and for their teachers? What kinds of adult math-
   emetics provision are being developed? How is this being translated into proc-
   cite and what provision is needed for developing teacher knowledge, skills and
 ­competence? Chapter 3 discusses all these issues in detail.
\textit{\\\\}
\begin{Schunk}
\begin{Sinput}
> R.cache::getCacheRootPath(defaultPrototype())
\end{Sinput}
\begin{Soutput}
[1] "/home/denis/.cache/R/R.cache"
\end{Soutput}
\end{Schunk}
\textit{\\\\}
\textbf{ 2.3 Current Paradoxes, Tensions and Potential Strategies }
\textit{\\\\}
 It would be wrong to say that there is full or uncomplicated consensus when it
 comes to the issues we grapple with in adult mathematics education. The research
 domain itself is not clearly defined. The discourse on how numeracy is concept-
 aliased and its relationship with mathematics and literacy is still a matter of 
 debate.
 There is tension between what policy makers define as numeracy and what is sub-
   subsequently implemented on the ground through the provision that is offered. 
 How
 can a community in Ireland, a community in South East Asia or a community in
 New Headland conceptualize numeracy and develop associated policy provision
 to meet the needs of their people? There is clearly no absolute measure, so how
 do we reconcile the multiplicity of interpretations in policy and provision? This
 part explores the paradoxes and tensions that exist in the research domain, practice
 and provision and offers some constructive recommendations to address the issues
 raised. Thinking about a good definition for the research domain and its bounds-
   rise is an important part of working in this area since we have started in 
 the 1990s.
\textit{\\\\}
\begin{Schunk}
\begin{Sinput}
> knitr::current_input(dir = ".")
\end{Sinput}
\begin{Soutput}
NULL
\end{Soutput}
\end{Schunk}
\textit{\\\\}
\textbf{ 2.3.1 The Disparate and Competing Conceptualization }
\textit{\\\\} 
 of Numeracy
 There have been many excellent reviews of the conceptualization of numeracy
 and its development since the 1990s (see for example Kaye 2010; Cob-en 2003;
 Gall 2009). In general terms the conceptualization of numeracy focuses around its
 relationship with both mathematics and literacy. Aguirre and O’Donoghue (2003)
 developed an organizing framework (Concept Sophistication in Numeracy—
 an Organizing Framework), which considers the development of the concept of
 numeracy as a continuum with three merging phases: Formative, Mathematical,
 and Integrative. The phases represent an incrementally-increasing degree of
 sophistication in conceptualization. Starting from a very limited concept of
 numeracy, where it is considered as basic arithmetic skills (formative phase), the
 framework then moves through to a concept of numeracy as being ‘mathematics
 in context’, which recognizes the importance of making explicit the significance
 of mathematics in daily life (Mathematical Phase). The continuum culminates in
 a conceptualization which views numeracy as a complex, multifaceted sophist-
   carted construct, incorporating, the mathematics, communication (incl. literacy),
 and cultural, social, emotional and personal aspects of each individual in context
 (Integrative Phase).
 Cob-en (2006) rightly points out that although conceptualization of numeracy
 always includes mathematics is does not work in reverse. Further she highlights
 how numeracy in some circumstances is conveyed as a component of ­mathematics
 e.g., Wedge et AL. (1999), and in others, how numeracy is considered to be “not
 less than maths but more” (Johnston and Tout 1995). Others have highlighted
\textit{\\\\}
\begin{Schunk}
\begin{Sinput}
> con <- DBI::dbConnect(RSQLite::SQLite(), ":memory:")
> DBI::dbWriteTable(con, "mtcars", mtcars)
> rs <- DBI::dbSendQuery(con, "SELECT * FROM mtcars")
> DBI::dbHasCompleted(rs)
\end{Sinput}
\begin{Soutput}
[1] FALSE
\end{Soutput}
\begin{Sinput}
> ret1 <- DBI::dbFetch(rs, 10)
> DBI::dbHasCompleted(rs)
\end{Sinput}
\begin{Soutput}
[1] FALSE
\end{Soutput}
\begin{Sinput}
> ret2 <- DBI::dbFetch(rs)
> DBI::dbHasCompleted(rs)
\end{Sinput}
\begin{Soutput}
[1] TRUE
\end{Soutput}
\begin{Sinput}
> DBI::dbClearResult(rs)
> DBI::dbDisconnect(con)
\end{Sinput}
\end{Schunk}
\textit{\\\\}
\textbf{ 2.3.1.1 Communication (1) (Fig. 2.1) }
\textit{\\\\}
 When an individual encounters a numeracy issue in a particular context, he or
 she perceives the situation from his/her own frame of reference. The individual’s
 frame of reference is a consequence of their life experiences and the consequent
 values, beliefs and attitudes. In a numeracy situation, the individual’s frame of
 reference in inextricably linked with their mathematical skills and knowledge and
 their communication (including literacy) skills. At this stage the individual is faced
 with the task of interpreting the information in whatever form it is communicated
 which describes the ‘issue’. The level of interpretation will be different for each
 individual and is determined by an individual’s facility with the particular content
 and form of communication and their ability to interpret that information.
\textit{\\\\}


\end{document}
